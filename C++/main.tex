\input{template.tex}

% Dokumentinformationen
\newcommand{\SUBJECT}{Zusammenfassung}
\newcommand{\TITLE}{C++}

\input{front.tex}

\lstset{style=cevelop-style}
\section{Trivia}


\subsection{Advance vs. Next}
\textbf{std::advance}

\begin{itemize}
\item modifies its argument
\item returns nothing
\item works on input iterators or better (or bi-directional iterators if a negative distance is given)
\end{itemize}

\noindent\textbf{std::next}

\begin{itemize}
\item leaves its argument unmodified
\item returns a copy of the argument, advanced by the specified amount
\item works on forward iterators or better (or bi-directional iterators if a negative distance is given))
\end{itemize}

\subsection{API}
\textbf{Vector} \\
- at, [ ], front, back, empty, size, clear, insert, erase, push\_back, pop\_back\\
\textbf{Set} \\
- empty, size, clear, insert, erase, count, find, contains\\
\textbf{Map} \\
- at, [ ], empty, size, clear, insert, erase, count, find, contains\\
\textbf{Multimap}\\
- empty, size, clear, insert, erase, count, find, contains


\subsection{Iterators}
\begin{lstlisting}[language=C++]
auto it1 = set.begin(); // std::set::const_iterator
auto it2 = string.crend(); // std::string::const_reverse_iterator
auto it3 = string.end(); // std::string::iterator
auto it4 = set.end(); // std::set::const_iterator
\end{lstlisting}
\begin{lstlisting}[language=C++]
vector<char> content{'S','t','a','c','k','o','v','e','r','f','l','o','w'};
auto it1 = begin(content); 
cout << *(++it1); // it1 inkrementieren, dann dealozieren => t
cout << ++(*it1); // it1 dealozieren, dann Buchstaben inkrementieren => ++S => T
\end{lstlisting}
\begin{lstlisting}[language=C++]
vector<char> content{'S','t','a','c','k','o','v','e','r','f','l','o','w'};
auto it1 = begin(content); 
++it1; // it1 zeigt auf 2. Position im vector
sort(begin(content), end(content));
std::cout << *it1; // it1 immer noch auf 2 Pos. => 'a' (sortiert)
\end{lstlisting}
\begin{lstlisting}[language=C++]
vector<char> content{'S','t','a','c','k','o','v','e','r','f','l','o','w'};
// Sacefkloortvw
auto it2 = remove(begin(content), end(content), 'o'); // Sacefklrtvwoo, it2 zeigt auf 1. 'o'
cout << distance(it2, end(content)); // 2
content.erase(it2); // loest nur 1. 'o', sonst muss von-bis angegeben werden
cout << content.size(); // 12
\end{lstlisting}

\subsection{Output with Copy}
\begin{lstlisting}[language=C++]
std::ostream_iterator<char> out{std::cout, "delmiter"};
std::copy(myset.begin(), myset.end(), out);
\end{lstlisting}
\subsection{Transform}
\begin{lstlisting}[language=C++]
std::transform(begin(counts), end(counts), begin(letters),
std::back_inserter(combined), [](int i, char c) {return std::string(i, c);});
\end{lstlisting}
\begin{lstlisting}[language=C++]
//transform over set with inserter
void test() {
	using namespace std;
	string const input("Test");
	using out = std::ostream_iterator<char>;
	set<char> s{};

	std::transform(begin(input), end(input), inserter(s, s.begin()),::toupper);
	copy(begin(s), end(s), out(cout, "-"));
}
\end{lstlisting}
\begin{lstlisting}[language=C++]
//transform over 2 iterators
using namespace std;
using out = ostream_iterator<string>;
transform(word.begin(), word.end(), values.begin(), out{cout, "\n"}, formatOutput);
\end{lstlisting}
\subsection{Accumulate}
\begin{lstlisting}[language=C++]
#include <algorithm>
#include <iterator>

transform(word.begin(), word.end(), back_inserter(values), toLetterValue);

using out = ostream_iterator<string>;

transform(word.begin(), word.end(), values.begin(), out{cout, "\n"}, formatOutput);

accumulate(values.begin(), values.end(), 0)
\end{lstlisting}
\subsection{Destructors (non-virtual) with virtual members are a design error}
\begin{lstlisting}[language=C++]
// Output:
// put into	trash
struct Fuel	{
		virtual void	burn()	=	0;
		/* virtual */ ~Fuel()	{	std::cout	<<	"put	into	trash\n";	}
};
struct Plutonium	:	Fuel	{
		void	burn()	{	std::cout	<<	"split	core\n";	}
		~Plutonium()	{	std::cout	<<	"store	many	years\n";	}
};
int	main()	{
		std::unique_ptr<Fuel>	surprise	=	std::make_unique<Plutonium>();
}
\end{lstlisting}
\subsection{Assignment through References copies into Original object}
\includegraphics[width=0.9\linewidth]{images/assignment.png}
\section{Histogram}
\begin{lstlisting}[language=C++]
#ifndef HISTOGRAM_H_
#define HISTOGRAM_H_
#include <map>

template<typename T>
struct Histogram {
    void insert (T const key) {
        ++m[key];
    }
    
    unsigned count (T const key) const {
        return m.find(key) != m.end() ? m[key] : 0u;
    }
    // oder
    unsigned count(T const key) const{
		auto result = myMap.find(key);
		if (result == myMap.end()) {
			return 0u;
		} else {
			return result->second;
		}
	}
    
private:
     std::map<T, unsigned> m{};
}
#endif

\end{lstlisting}
\pagebreak
\subsection{HistoramEntry}

\begin{lstlisting}[language=C++]
#ifndef HISTOGRAMENTRY_H_
#define HISTOGRAMENTRY_H_
#include "Word.h"
#include <algorithm>
#include <boost/operators.hpp>
struct HistogramEntry :boost::less_than_comparable<HistogramEntry>, boost::equality_comparable<HistogramEntry> {
	HistogramEntry (Word w, int amount);
	
	inline bool operator <(HistogramEntry const & lhs, HistogramEntry const & rhs) {
		return lhs.amount > rhs.amount;
	}

	inline bool operator >(HistogramEntry const & lhs, HistogramEntry const & rhs) {
		return lhs.amount < rhs.amount;
	}

	inline bool operator ==(HistogramEntry const & lhs, HistogramEntry const & rhs)  {
		return lhs.amount == rhs.amount && lhs.word == rhs.word;
	}
	inline std::ostream& operator<<(std::ostream &out, HistogramEntry const & histogram) {
		histogram.word.print(out);
		out << ": " << histogram.amount;
		return out;
	}
private:
	Word word;
	int amount;
};
#endif /* HISTOGRAMENTRY_H_ */
\end{lstlisting}
\section{Word}
\begin{lstlisting}[language=C++]
#ifndef	WORD_H_
#define	WORD_H_

#include <algorithm>
#include <cctype>
#include <iterator>
#include <string>
#include <ostream>

namespace text {

struct Word {
	Word();
	explicit Word(std::string const & value);
	void read(std::istream &is);
	void print(std::ostream & os) const;

	inline bool operator <(Word const & rhs) const {
		return std::lexicographical_compare(
			std::begin(this->value), std::end(this->value),
			std::begin(rhs.value), std::end(rhs.value),
			[](const char l, const char r) {
				return std::tolower(l) < std::tolower(r);
			}
		);
	}

	inline bool operator ==(Word const & rhs) const {
		return std::equal(
			std::begin(this->value), std::end(this->value),
			std::begin(rhs.value), std::end(rhs.value),
			[](const char l, const char r) {
				return std::tolower(l) == std::tolower(r);
			}
		);
	}

	bool operator>(Word const & other) const {
		return (other < *this);
	}

	bool operator<=(Word const & other) const {
		return !(other < *this);
	}

	bool operator>=(Word const & other) const {
		return !(*this < other);
	}

	bool operator!=(Word const & other) const {
		return !(*this == other);
	}
private:
	std::string value;
	bool isValid(std::string const & value);
};

inline std::istream & operator>>(std::istream & in, Word & word) {
	word.read(in);
	return in;
}

inline std::ostream& operator<<(std::ostream &out, Word const &word) {
	word.print(out);
	return out;
}
}

#endif	/*	WORD_H_	*/
\end{lstlisting}
\begin{lstlisting}[language=C++]
#include "word.h"

#include <iterator>
#include <algorithm>
#include <cctype>
#include <stdexcept>
#include <string>

using text::Word;

Word::Word() : value{"default"} {
}

void Word::read(std::istream &is) {
	if(is.good()) {
		using iter = std::istreambuf_iterator<char>;
		iter input{is};
		iter eof{};
		auto firstChar = std::find_if(input, eof, ::isalpha);
		std::string readWord{};
		std::find_if(firstChar, eof, [&readWord](char c) {
			bool isWordFinished {!std::isalpha(c)};
			if (!isWordFinished) {
				readWord += c;
			}
			return isWordFinished;
		});
		if (isValid(readWord)) {
			value = readWord;
		} else {
			is.setstate(std::ios_base::failbit);
		}
	}
}

Word::Word(std::string const & value) : value { value } {
	if (!isValid(value)) {
		throw std::invalid_argument{"Word isn't valid"};
	}
}

bool Word::isValid(std::string const & value) {
	return !value.empty() && std::all_of(std::begin(value), std::end(value), ::isalpha);
}

void Word::print(std::ostream & os) const {
	os << value;
}
\end{lstlisting}

\section{ENUM}
\begin{lstlisting}[language=C++]
namespace calendar {
    enum class DayOfWeek {
        Mon, Tue, Wed, Thu, Fri, Sat, Sun
    };
}
bool is_weekend(calendar::DayOfWeek day) {
    return day == calendar::DayOfWeek::Sat ||
        day == calendar::DayOfWeek::Sun;
}
\end{lstlisting}

\includegraphics[width=0.9\linewidth]{images/enum.png}
\section{Vectorset}
\begin{lstlisting}[language=C++]
#ifndef VECTORSET_H_
#define VECTORSET_H_
#include <vector>
#include <set>
#include <functional>
#include <algorithm>

template <typename T, typename COMPARE=std::less<T>>
struct vectorset : public std::vector<T> {

   using vectorType = std::vector<T>;
   using vectorType::vectorType;

// Aliases
   using size_type = typename vectorType::size_type;
   using reference = typename vectorType::reference;
   using const_reference = typename vectorType::const_reference;
   using iterator = typename vectorType::iterator;
   using const_iterator = typename vectorType::const_iterator;
   
    vectorset() = default;

    explicit vectorset(std::initializer_list<T> li) : vectorType{li} {
        std::sort(this->begin(), this->end(), COMPARE());
    }

    template <typename ITER>
    vectorset(ITER b, ITER e) : vectorType(b, e) {
       std::sort(this->begin(), this->end(), COMPARE());
    }

    template <typename Elt>
    explicit operator std::multiset<Elt>() const{
    	return std::multiset<Elt>(this->begin(), this->end());
    }

// Functions
   const_iterator find(T const key) const {
        return std::find_if(this->cbegin(), this->cend(), [&key](const T &entry) {
                COMPARE comp{};
                return !comp(key, entry) && !comp(entry, key);
            });
        // return std::find_if(this->cbegin(), this->cend(), [](const T &e) { return e == key; });
        // is equivalent to: return std::find(this->cbegin(), this->cend(), key)
   }

   size_type count(T const key) const {
        return std::count_if(this->cbegin(), this->cend(), [&key](const T &entry) {
                COMPARE comp{};
                return !comp(key, entry) && !comp(entry, key);
        });
   }

   std::multiset<T, COMPARE> asMultiset() {
        return std::multiset<T, COMPARE> (this->cbegin(), this->cend());
   }
};
#endif
\end{lstlisting}

\section{Indexable Set}
\begin{lstlisting}[language=C++]
#ifndef INDEXABLESET_H_
#define INDEXABLESET_H_

#include <set>
#include <stdexcept>
#include <algorithm>

template<typename T, typename COMPARE=std::less<T>>
struct indexableSet	: std::set<T, COMPARE> {
	using container	= std::set<T, COMPARE>;
	using container::container;
	using difference_type = typename container::difference_type;
	using const_reference = typename container::const_reference;

	const_reference	at(difference_type index) const {
		if (index < 0) {
			const long long unsigned int absIndex = abs(index);
			if (absIndex > this->size()) {
				throw std::out_of_range(absIndex + " is out of range");
			}
			return *std::prev(this->cend(), absIndex);
		} else {
			if (static_cast<long long unsigned int>(index) >= this->size()) {
				throw std::out_of_range(index + " is out of range");
			}
			return *std::next(this->cbegin(), index);
		}
	}

	const_reference	operator[](difference_type index) const {
		return this->at(index);
	}

	const_reference	front() const {
		return this->at(0);
	}

	const_reference	back() const {
		return this->at(-1);
	}
};

#endif /* INDEXABLESET_H_ */
\end{lstlisting}

\section{Deck}
\begin{lstlisting}[language=C++]
#ifndef DECK_H
#define DECK_H
#include <deque>
#include <algorithm>
#include <stdexcept>
#include <random>
#include <iterator>

template <typename T>
class Deck {
    using container = std::deque<T>;
    container c;
    using size_type = typename container::size_type;
    using const_iterator = typename container::const_iterator;
    using const_reverse_iterator = typename container::const_reverse_iterator;
    using const_reference = typename container::const_reference;
public:
    Deck() = default;
    explicit Deck(std::initializer_list<T> li) : c{li} {
        this->shuffle();
    }
    template <typename ITER>
    Deck(ITER b, ITER e) : c(b, e) {
        this->shuffle();
    }

    size_type size() const { return c.size(); }
    bool empty() const { return c.empty(); }
    const_reference front() const {
        checkContainer();
        return c.front();
    }

    const_reference back() const {
        checkContainer();
        return c.back();
    }

    void push_back(T const elem) {
        c.push_back(elem);
        shuffle();
    }

    void pop_front() {
        checkContainer();
        c.pop_front();
    }

    void shuffle() {
        std::random_device rd;
        std::mt19937 g(rd());
        std::shuffle(c.begin(), c.end(), g);
    }

    void checkContainer() const {
        if (c.empty()) {
            throw std::out_of_range{"Out of range"};
        }
    }

    // Iteratoren
    const_iterator begin() const { return c.begin(); }
    const_iterator cbegin() const { return c.cbegin(); }

    const_iterator end() const { return c.end(); }
    const_iterator cend() const { return c.cend(); }

    const_reverse_iterator rbegin() const { return c.rbegin(); }
    const_reverse_iterator crbegin() const { return c.crbegin(); }

    const_reverse_iterator rend() const { return c.rend(); }
    const_reverse_iterator crend() const { return c.crend(); }
};
#endif
\end{lstlisting}
\section{Sack}

\subsection{Iterator constructors}
\begin{lstlisting}[language=C++]
void	createSackFromIterators()	{
		std::vector	values{3,	1,	4,	1,	5,	9,	2,	6};
		Sack<int>	aSack{begin(values),	end(values)};
		ASSERT_EQUAL(values.size(),	aSack.size());
}

template	<typename T>
class Sack	{
//...
public:
		template	<typename Iter>
		Sack(Iter	begin,	Iter	end)	:	theSack(begin,	end)	{}
		//...
};
\end{lstlisting}

\begin{lstlisting}[language=C++]
// Retain default constructor
Sack() = default;
\end{lstlisting}

\subsection{Initializer list constructors}

\begin{lstlisting}[language=C++]
Sack(std::initializer_list<T>	values)	: theSack(values)	{}
\end{lstlisting}

\subsection{Extracting a std::vector}
\subsubsection{Usage}
\begin{lstlisting}[language=C++]
//explicit conversion operator
Sack<unsigned>	aSack{1,	2,	3};
auto	values	=	static_cast<std::vector<unsigned>>(aSack);
\end{lstlisting}

\begin{lstlisting}[language=C++]
//member function
Sack<unsigned>	aSack{1,	2,	3};
auto	values	=	aSack.asVector();
auto	doubleValues	=	aSack.asVector<double>();
\end{lstlisting}

\subsubsection{Implementation}

\begin{lstlisting}[language=C++]
//explicit conversion operator
template	<typename Elt>
explicit	operator	std::vector<Elt>()	const	{
		return	std::vector<Elt>(begin(theSack),	end(theSack));
}
\end{lstlisting}

\begin{lstlisting}[language=C++]
//member function
template	<typename Elt	=	T>
auto	asVector()	const	{
		return	std::vector<Elt>(begin(theSack),	end(theSack));
}
\end{lstlisting}

\subsection{Deduction Guide }

\begin{lstlisting}[language=C++]
template	<typename	Iter>
Sack(Iter	begin,	Iter	end)	->	Sack<typename	std::iterator_traits<Iter>::value_type>;
\end{lstlisting}

\subsection{Template Specialization}
\begin{lstlisting}[language=C++]
//Partial Specialization
template <typename T>
struct Sack<T *>;
// Explicit Specialization
template <>
struct Sack<char const *>;
\end{lstlisting}


\section{Espresso}
\subsection{espresso.h}
\begin{lstlisting}[language=C++]
#ifndef ESPRESSO_H_
#define ESPRESSO_H_

#include <iosfwd>
namespace Coffee {

enum class Aroma {
	Cosi, Dharkan, Fortissio, Kazaar, Livanto, Water
};

struct Espresso {
	Aroma aroma{Aroma::Water};

	Espresso() = default;
	explicit Espresso (Aroma aroma);

	bool operator==(Espresso const &other) const;
	bool operator!=(Espresso const &other) const;
	friend std::istream & operator>>(std::istream & in, Espresso & e);


};
}
#endif /* ESPRESSO_H_ */
\end{lstlisting}
\pagebreak
\subsection{espresso.cpp}
\begin{lstlisting}[language=C++]
#include <string>
#include <istream>
#include <map>
#include "espresso.h"

namespace Coffee {

using namespace std::string_literals;

std::map<std::string, Aroma> const aromaNames {
	{"Cosi"s, Aroma::Cosi},{"Dharkan"s, Aroma::Dharkan},
	{"Fortissio"s, Aroma::Fortissio},{"Kazaar"s, Aroma::Kazaar},
	{"Livanto"s, Aroma::Livanto},{"Water"s, Aroma::Water},
};

Espresso::Espresso(Aroma aroma)  : aroma{aroma}{};

bool Espresso::operator ==(Espresso const &other) const {
	return this->aroma == other.aroma;
}

bool Espresso::operator !=(Espresso const &other) const {
	return !(*this == other);
}

std::istream & operator>>(std::istream & in, Espresso & e){
	std::string stringAroma{};
	if (in >> stringAroma) {
		auto const aroma = aromaNames.find(stringAroma);
		if (aroma == aromaNames.end()){
			in.setstate(std::ios_base::failbit);
			return in;
		} else {
			e = Espresso{aroma->second}; // otherwise: e.aroma = aroma->second; 
			return in;
		}
	} else {
		return in;
	}
}
}
\end{lstlisting}

\section{Functor}
\begin{lstlisting}[language=C++]
#include <iostream>
#include <string>
#include <set>
#include <cctype>
#include <iterator>
#include <algorithm>

struct caselessless {
	bool operator()(char const c1, char const c2) const {
		return std::tolower(c1) < std::tolower(c2);
	}
};

void teilB(){
	using namespace std;
	string kasten("OachkatzlSchwoaf");

	set<char, caselessless> s{};

	for (char c : kasten) s.insert(c);

	cout << s.size() << '\n';

	using out = std::ostream_iterator<char>;

	copy(s.begin(), s.end(), out(cout, "."));
}
\end{lstlisting}

\input{appendix.tex}