\section{OS API}
\subsection{Aufgaben OS}
\begin{itemize}
    \item Abstraktion/Portabilität
    \item Resourcenmanagement/Isolation
    \item Benutzerverwaltung/Sicherheit
\end{itemize}
\subsection{Prozessor Privilege Level}
mind. 2 Privilege Lvls.: Kernel Mode, User Mode\\
Kernel bestimmt in welchem Modus ein Programm läuft (Entscheid somit softwareseitig)
%\textbf{Micro-Kernel:} selbst Gerätetreiber in User Mode \textcolor{green}{+} Stabilität, Analysierbarkeit \textcolor{red}{-} Performance (häufige Mode-Wechsel)
%\textbf{Monolithisch (Linux etc.):} sehr viel Funkt. in Kernel, welche nicht nötig wären \textcolor{green}{+} Performance (weniger Wechsel) \textcolor{red}{-}Programmierfehler (wilde Pointer, einige Anwendungen in Kernel, Abstimmung zwischen Programmierern etc.)

\subsubsection{Wechsel vom User Mode in Kernel Mode}
\textit{syscall $\rightarrow$ Kernel-Mode $\rightarrow$ Instr. Pointer auf Call Handler\\}
Linux-Kernel nicht binärkompatibel

\subsection{Programmargumente}
Argumente vom OS in Speicherbereich des Programms als Array mit Pointern auf null-terminierte Strings\\
\prgc{main (int argc, char** argv)}: \prgc{argc} Anz. Argumente, \prgc{argv} Pointer auf Array mit Strings(\prgc{char*}), \prgc{argv[0]} Programmname!

\subsection{Umgebungsvariablen}
Umgebungsvar. vom OS in Speicherbereich des Programms kopiert als \textcolor{green}{Array mit Pointern} auf \textcolor{brown}{null-terminierte Strings} (wie Programmarg.)\\
POSIX jeder Prozess eigene Umgebungsvariablen\\
\textcolor{blue}{Key} (unique) \textcolor{red}{Value}\\
Umgebungsvariablen initial vom erzeugenden Prozess festgelegt (z.B. shell)

\prgc{putenv} ersetzt mit Pointer, keine Kopie (wie set)!