\section{Bibliotheken}

%\subsection{Statische Bibliotheken} 
%Archiv von Obj-Dateien, wie mehrere Obj-Dateien behandelt lib$<$name$>$.a\\
%\textcolor{green}{+}einfache Verwendung/Implementation
%\textcolor{red}{-}Neuerstellung der Programme bei Änderungen der Lib nötig
%\textcolor{red}{-}fixe Funktionalität

%\subsection{Dynamische Bibliotheken}
%Executable nur noch Referenz auf Bibliothek, zur Lade-/Laufzeit gelinkt
%\textcolor{green}{+}Programm muss nur Bibliothek laden, die es braucht

%\subsection{API}
%\begin{minted}{c}
%//Öffnet dynamische Bibliothek, Handle zurück
%void* dlopen(char* filename, int mode)
%//dlsym Adresse des Symbols als void*
%typedef int (*funct_t)(int);
%funct_t f = dlsym(handle, "my_function");
%int *i = dlsym(handle, "my_int");
%(*f)(*i);
%int dlclose(void* handle);
%char dlerror();
%\end{minted}

%\subsection{Shared Objects}
%Referenz in Executable nötig, OS sucht bei Programmstart automatisch richtige Bibliotheken\\
%Versionen und Unterversionen können gleichzeitig verwendet werden/existieren

\begin{tabular}{|l|l|l|}
\hline
   Lin.-Name:  & lib + Biblio. + .so & libmylib.so\\
   \hline
   SO-Name: & Lin.-Name + . + V.nr. & libmylib.so.2\\
   \hline
   Real-Name: & SO-Name + . + SubV.nr. & libmylib.so.2.1\\
   \hline
\end{tabular}
\prgc{/usr/lib}