\section{Shared Memory}
Frames des Hauptspeichers werden zwei Prozessen freigegeben:
\begin{itemize}
    \item In P1 wird Page V1 auf einen Frame F abgebildet
    \item In P2 wird Page V2 auf \underline{denselben} Frame F abgebildet
\end{itemize}
\textbf{Verwendung von Pointern: }Pointer relativ zu einer Anfangsadresse sein

%\subsection{POSIX API}
%OS benötigt 2 Typen von Objekten:
%\begin{itemize}
%    \item spezielles Objekt(wie Datei behandelt) zur Verwaltung von Infos zu gemeinsamem Speicher
%    \item Objekt pro Prozess um Mappings zu speichern
%\end{itemize}

%\begin{minted}{C}
%int fd = shm_open(//gibt Deskriptor zurück
%"/ mysharedmemory ",//name
%O_RDWR | O_CREATE, //zum lesen/schreiben öffnen 
%S_IRUSR | S_IWUSR);//Berechtigung Lesen/Schreiben

%int ftruncate ( int fd , offset_t length );
%//Grösse setzen, zwingend!!
%int close ( int fd ); 
%//Shr bleibt in System, auch ohne Prozess
%int shm_unlink (const char * name);//löschen

%//Mappt Shr in virt. Adressraum des akt. Prozesses
%//gibt Adresse des 1. Bytes zurück
%void * address = mmap (
%0 , // void * hint_address ,
%size_of_shared_memory , // size_t length ,
%PROT_READ | PROT_WRITE , // int protection ,
%MAP_SHARED , // int flags ,
%fd , // int file_descriptor ,
%0 // off_t offset
%);

%int munmap ( void * address , size_t length );
%//entfernt Mapping
%\end{minted}

%\subsection{Vergleich Message-Passing/Shared Memory}
%\textbf{Shared Memory: }\textcolor{green}{+} schneller zu realisieren (Umwandlung Appl. mit 1 Prozess auf mehrere)
%\textcolor{red}{-}Programme schwer wartbar \textcolor{red}{-}implizite Abhängigkeiten$\rightarrow$nebenläufig, aber nicht echt parallel
%\textcolor{red}{-}System weniger stark modularisiert, Prozesse schlechter gegeneinander geschützt
%\textbf{Message Passing: }\textcolor{red}{-}mehr Engineering-Aufwand (bei bestehenden Applikationen, neu Implementation nötig) \textcolor{green}{+}bei sauber gekapselten Programmen geringeres Problem \textcolor{green}{+}Appl. leicht ausbaufähig zu verteilten Systemen (bei einigen OS schon integriert)\\
%\subsection{Performance-Vergleich}\\
%\begin{description}
%    \item[Einzel-Prozessor-Systeme] Shared Memory
%    \item[Mehr-Prozessor-Systeme] SMR benötigt zusätzlichen Aufwand hardwaremässig
%    \begin{itemize}
%        \item Änderung im Speicher, häufig erstmals nur im Cache
%        \item Caches der anderen Prozis sehen Änderung nicht
%        \item $\rightarrow$Caches nicht kohärent
%        \item Hardware muss dies updaten (Cache Coherency Protocol)
%    \end{itemize}
%\end{description}
$\rightarrow$beide Varianten liegen bei Mehr-Prozessoren-Systemen gleichauf, Message-Passing vermutlich perfomanter in Zukunft


