\section{Interprozess-Kommunikation (IPC)}
\subsection{Signale}
ermöglichen Unterbruch eines Prozesses von aussen\\
wird vom OS wie ein Interrupt behandelt
%\begin{itemize}
%    \item Unterbruch des laufenden Prozesses/(Threads)
%    \item Auswahl Signal-Handler Funktion
%    \item Ausführen Signal-Handler Funktion
%    \item Fortsetzen des Prozesses (falls Prozess nicht beendet)
%\end{itemize}

\subsubsection{Quelle von Signalen}
\textbf{Hardware/OS: }Ungültige Instruktion, (\textit{segmentation fault})
\textbf{Andere Prozesse: }Ctrl-C

\subsubsection{Signale behandeln}
\textbf{Ausser SIGKILL \& SIGSTOP alle Handler überschreibbar}

\subsubsection{Wichtige Signale}
SIGTERM bittet Programm zu beenden \\
SIGKILL killt Programm (SIGKILL/SIGSTOP nicht überschreibbar)

%\bash{kill 1234 5678} SIGTERM an Prozesse 1234 5678\\
%\bash{kill -Kill 1234} SIGKILL an Prozess 1234\\
%\bash{kill -l} listet alle möglichen Signale auf 
%\begin{minted}{C}
%sigemptyset(sigset_t *set);
%sigfillset(sigset_t *set);
%sigaddset(sigset_t *set, int signal);//hinzufügen
%sigdelset(sigset_t *set, int signal);//entfernen
%sigismember(const sigset_t *set, int signal);//1 -> true
%\end{minted}

\subsection{Message-Passing}
%Implementationsabhängig: fixe/variable Nachrichtengrösse + folgende Punkte: %, Direkte/indirekte Kommunikation, Synchrone/asynchrone Kommunikation, Pufferung, Mit/ohne Prioritäten

\subsubsection{Direkte Kommunikation}
Sender muss Empfänger kennen\\
symmetrisches: kennt vs. Asymmetrisches: erhält ID

\subsubsection{Indirekte Kommunikation}
Mailbox/Port/Queue kennen\\
Queue gehört zu Prozess oder zu OS(Lösch/Erzeugmechanism.)

\subsubsection{Synchronisation}
blockierend (synchron)/nicht-blockierend(asynchron)\\
%\textbf{asynchrones Empfangen} 
$\rightarrow$Alle Kombinationen möglich (z.B. synchroner Sender/asynchroner Receiver)

\subsubsection{Rendezvous}
Sender \& Empfänger blockierend\\
OS kann direkt vom Sende- in Empfängerprozess kopieren (meistens ungepuffert)
(implizite Synch, Impl. Producer/Consumer-Problem)

%\subsubsection{Pufferung}
%\textit{Keine: }Queue-Länge = 0 | Sender muss blockieren
%\textit{Beschränkt: }$n$ Nachrichten können zwischengespeichert werden | Sender blockiert, wenn Queue voll
%\textit{Unbeschränkt}

%\subsubsection{Prioriäten}
%Empfänger holt Nachricht mit höchster Prio zuerst

\subsection{POSIX API}
\begin{itemize}
    \item Message-Queues vom OS
    \item variable Nachrichtenlänge, Maximum pro Queue einstellbar
    \item synchrone/asynchrone Verwendung
    \item Prioritäten
\end{itemize}
Sockets: bind, listen, accept, recv, send, close

